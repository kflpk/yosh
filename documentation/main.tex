\documentclass{article}
\usepackage[utf8]{inputenc}
\usepackage[OT4]{fontenc}
\usepackage{polski}
% \usepackage{lmodern}
\usepackage[colorlinks=true, allcolors=blue]{hyperref}
\usepackage[a4paper, margin={2.54cm, 1.54cm}]{geometry}
\usepackage{graphicx}
\usepackage{titlesec}
\usepackage{titling}
\usepackage{enumitem}

\titleformat{\section}
{\LARGE \bfseries}
{\thesection.}
{0.5em}
{}[\titlerule]

\titleformat{\subsection}
{\Large \bfseries}
{\thesubsection.}
{0.5em}
{}[]

% \renewcommand{\labelenumii}{\theenumi.\theenumii.}
\setlist[enumerate]{label*=\arabic*.}


\graphicspath{ {./img/} }

% Title page sourced from https://www.overleaf.com/latex/templates/agh-praca-inzynierska-magisterska/mmbvwfgrpzby
% By Robert Gałat
% Licenced under Creative Commons CC BY 4.0

\newcommand{\setSupervisor}[1]{
    \newcommand{\supervisor}{#1}
}
\newcommand{\setFaculty}[1]{
    \newcommand{\faculty}{#1}
}
\newcommand{\setMajor}[1]{
    \newcommand{\major}{#1}
}

\newcommand{\setDepartment}[1]{
    \newcommand{\department}{#1}
}

\newcommand{\setReviewer}[1]{
    \newcommand{\reviewer}{#1}
}

\newcommand{\setGroup}[1]{
    \newcommand{\group}{#1}
}

\makeatletter
\renewcommand{\maketitle}{\begin{titlepage}
        \center{\includegraphics[height=37.5mm]{agh_nzw_s_pl_1w_wbr_rgb_150ppi}\\}
        \rule{30mm}{0pt}
        % {\large \textsf{\department}}\\
        % \rule{\textwidth}{3pt}\\
        \rule[2ex]
        {\textwidth}{1pt}\\
        \vspace{7ex}
        \begin{center}
            {\LARGE \textsf{Dokumentacja do projektu}}\\
            \vspace{8ex}
            %% ------------------------ TYTUL PRACY --------------------------------------
            {\bf \huge \textsf{\@title}}\\
            \vspace{14ex}
            %% ------------------------ PRZEDMIOT ----------------------------------------
            {\Large \textsf{z przedmiotu}}\\
            \vspace{3ex}
            {\bf \huge \textsf{Programowanie obiektowe}}\\
            \vspace{3ex}
            {\sf \large Kierunek:} {\large \textsf{\faculty}}\\
            \vspace{10ex}
            % --------------------------- IMIE I~NAZWISKO -------------------------------
            {\it \Large \textsf{\@author}}\\
            \vspace{3ex}
            {\it \large \textsf{\group}}\\
            \vspace{10ex}
            %% ------------------------ OPIEKUN PRACY ------------------------------------
            {\small Prowadzący: \textsf{\supervisor}}\\
            \vspace{12ex}
            {\small \textsf{Kraków, \@date}}
        \end{center}
    \end{titlepage}%
}
\makeatother

\setDepartment{Elektronika II}
\setGroup{piątek 13:00}
\setFaculty{Elektronika II}
\setReviewer{Rafał Frączek}
\setSupervisor{Rafał Frączek}

\title{Command line interpreter}
\author{Kacper Filipek}
\date{\today}

\begin{document}

\fontfamily{cmss}
\selectfont

\maketitle

\newpage

\section{Opis projektu}

Program to autorska implementacja powłoki systemowej na systemy z rodziny Unix. Implementuje funkcje wywoływania programów poprzez
komenty oraz ustawianie własnego znaku zachęty.

\section{Project description}

This program is an original implementation of a Unix shell. It implements functions such as executing programs by 
a command and setting a custom command prompt.

\section{Instrukcja użytkownika}

\subsection*{Funkcje wbudowane}
Program zawiera następujace funkcje wbudowane, które mozna wywołać z wiersza poleceń:
\begin{itemize}
    \item \texttt{cd} - zmienia aktualny katalog\\
    Używanie: \texttt{cd path}, gdzie \texttt{path} to ścieżka do katalogu, relatywna lub absolutna
    \item \texttt{vars} - wyświetla zmienne lokalne dla aktualnej sesji powłoki
    \item \texttt{welcome} - wyświetla wiadomość powitalną z podstawowymi informacjami
    \item \texttt{help} - wyświetla pomoc z omówieniem najważniejszych funkcji
    \item \texttt{exit} - wychodzi z programu i zapisuje historię komend do pliku
\end{itemize}

\subsection*{Wywoływanie programów}
Podstawową funckją programu jest wywoływanie programów poprzez interfejs wiersza poleceń. Aby wywołać program, należy
wpisać nazwę jego pliku wykonywalnego oraz listę argumentów. Program zostanie wywołany, jeśli znajduje się w katalogu
ze zmiennej PATH. Można również wywoływać programy podając ich ścieżkę absolutną lub relatywną. 
Przykładowe użycia:
\begin{itemize}
    \item \texttt{ls -lah} - wywołanie ze zmiennej PATH
    \item \texttt{/usr/sbin/usermod -a -G sudo username} - wywołanie poprzez ścieżkę absolutną
    \item \texttt{./my\_script.sh} - wywołanie poprzez ścieżkę relatywną
\end{itemize}

Możliwe jest przekazanie argumentów ze znakami białymi przy pomocy cudzysłowów. Ciągi znaków objęte cudzysłowem zostaną
w całości przekazane do programu jako jeden argument. Przykładowo:
\begin{itemize}
    \item \texttt{git commit -m "krotki opis commita"}
    \item \texttt{cat "nazwa pliku ze spacjami.txt"}
\end{itemize}


\subsection*{Zmienne lokalne}
Program zawiera podstawowe wsparcie dla zmiennych lokalnych. Zmienną przypisuje się przy pomocy znaku równości 
w następujący sposów.\\
\texttt{NAME=VALUE}, gdzie NAME to nazwa zmiennej, a VALUE to jej wartość. Nazwa zmiennej musi spełniać następują założenia:
\begin{itemize}
    \item Składać się z przynajmniej jednego znaku
    \item Zaczynać się od litery lub znaku podkreślenia (\_).
    \item Następne znaki mogą być literami, podkreśleniami, lub cyframi
\end{itemize}
Wokół znaku równości nie mogą znajdować się żadne znaki białe. Podobnie jak w wywyływaniu możemy użyć cudzysłowów, aby
przekazać większą liczbę znaków jako wartość, np: \texttt{ZMIENNA="Litwo, ojczyzno moja!"}.

\subsection*{Zmienne środowiskowe}
Program umożliwia ustawianie zmiennych środowiskowych. Ich przypisywanie odbywa się w sposób podobny do zmiennych lokalnych,
z tym że przypisanie poprzedzone jest słowem kluczowym "export". 
Przykładowo: \texttt{export XDG\_CONFIG\_HOME="/home/user/.config"}

\subsection*{Znak zachęty}
W programie możliwe jest zdefiniowanie własnego znaku zachęty poprzez modyfikację zmiennej lokalnej \texttt{PROMPT}.
Zdefiniowane są specjalne sekwencje, pozwalające wyświetlić pewne użyteczne informacje w znaku zachęty.

Lista sekwencji specjalnych
\begin{itemize}
    \item \texttt{\%u} - nazwa użytkownika
    \item \texttt{\%h} - nazwa hosta
    \item \texttt{\%w} - aktualny katalog
\end{itemize}

Oprócz tego możliwe jest używanie kolorów przez poniższe sekwencje. Tekst następujący
po sekwencji będzie miał odpowiadający jej kolor.
\begin{itemize}
    \item \texttt{\%\{red\}} - czerwony
    \item \texttt{\%\{green\}} - zielony
    \item \texttt{\%\{yellow\}} - żółty
    \item \texttt{\%\{blue\}} - niebieski
    \item \texttt{\%\{magenta\}} - magneta
    \item \texttt{\%\{cyan\}} - cyjan
    \item \texttt{\%\{white\}} - biały
    \item \texttt{\%\{default\}} - domyślny kolor tekstu w terminalu
\end{itemize}

\subsection*{Plik konfiguracyjny}
Program posiada plik konfiguracyjny zlokalizowany po ścieżką \texttt{.config/yosh/yoshrc}, który jest interpretowany
linijka po linijce na starcie programu. Można w nim zdefiniować zmienne środowiskowe i lokalne lub wywoływać programy.

\section{Kompilacja}

Program został napisany na systemy operacyjne z rodziny Linux i nie jest możliwe skompilowanie go na system Windows,
ponieważ wymaga on funkcji z pliku nagłówkowego \texttt{unistd.h}, takich jak \texttt{fork()} lub \texttt{waitpid()}. 
Prawdopodobnie kompilacja programu na systemy z rodziny BSD jest możliwa, ponieważ jest on zgodny ze standardem POSIZX, 
jednak program nie był na nich testowany.\\
Program można skompilować w następujący sposób.
    \begin{enumerate}
            \item Utworzyć folder \texttt{build/} poleceniem \texttt{mkdir build}
            \item Wejść do folderu  \texttt{build/}
            \item Wykonać polecenie \texttt{cmake .. \&\& make}
    \end{enumerate}

Po zbudowaniu plik wykonywalny będzie pod ścieżką \texttt{build/yosh}.

\section{Pliki źródłowe}

Projekt składa się z następujących plików źródłowych:
    \begin{itemize}
    \item \texttt{parser.hpp}, \texttt{parser.cpp} – deklaracja oraz implementacja klasy Parser
        \begin{itemize}
            \item \texttt{Parser()} - konstruktor klasy
            \item \texttt{Command parse(std:string command)} - funkcja parsująca polecenie w stringu na dedykowaną strukturę
            \item \texttt{char** parse\_to\_cstrings(std::vector<std::string> args)} - funkcja konwertująca wektor stringów na
            tablicę c-stringów
        \end{itemize}
    \item \texttt{prompt.hpp}, \texttt{prompt.cpp} - deklaracja oraz implementacja klasy Prompt
        \begin{itemize}
            \item \texttt{void display()} - wypisuje znak zachęty do wyjścia standardowego
            \item \texttt{void set\_prompt()} - setter znaku zachęty
        \end{itemize}
    \item \texttt{shell.hpp}, \texttt{shell.cpp} - deklaracja oraz implementacja klasy Shell
        \begin{itemize}
            \item \texttt{unsigned int execute\_command(std::string command)} - wykonuje polecenie zapisane jako string
            \item \texttt{void init()} - inicjuje stan powłoki
            \item \texttt{void loop()} - główna pętla powłoki
        \end{itemize}
    \item \texttt{utils.hpp}, \texttt{utils.cpp} - deklaracje i implementacje funkcji pomocniczych
    \item \texttt{main.cpp} – główny plik z implementacją funkcji main.
    \end{itemize}

\section{Zależności}
Program korzysta z następujących programów do kompilacji.
\begin{itemize}
    \item GNU make - program do automatyzacji kompilacji programu
    \item CMake - generator projektów do C++
\end{itemize}
Obie powyższe są dostępne w domyślnych repozytoriach większości dystrybucji Linuxa, jak i w systemach z rodziny BSD.

\section{Opis klas}

W projekcie utworzono następujące klasy:
\begin{itemize}
    \item \texttt{Parser} - klasa parsująca komendy
    \item \texttt{Prompt} - klasa implementująca znak zachęty
    \item \texttt{Shell}  - klasa implementująca główne elementy powłoki
\end{itemize}
     
\section{Zasoby}
Program używa dwóch plików do działania.
\begin{itemize}
    \item \texttt{.config/yosh/yoshrc} - plik zawierający komendy, które automatycznie są wykonywane przed startem 
    programu. 
    \item \texttt{.cache/yosh/history} - plik zawierający historię wyołanych poleceń. Jeśli plik nie istnieje, to
    program sam go stworzy i zapisze do niego historię.
\end{itemize}


\section{Dalszy rozwój i ulepszenia}
Program jest bardzo podstawową implementacją powłoki, więc istnieje duże pole do jego rozbudowy o dodatkową
funkcjonalność. Funkcjami, do dalszej implementacji mogą być:
\begin{itemize}
    \item przekierowywanie wyjśćia programu
    \item przekazywanie danych między programami (pipe)
    \item odnoszenie się do zmiennych lokalnych
    \item przeszukiwanie historii
    \item podstawowe wsparcie dla pisania skryptów
\end{itemize}

\section{Inne}

Z uwagi na konieczność korzystania z API napisanego w języku C do wchodzenia w interakcji z systemem operacyjnym,
program w niektórych miejscach używa dynamicznej alokacji i wskaźników z C w celu kompatybilności z funkcjami
z nagłówka \texttt{unistd.h}.

\end{document}
